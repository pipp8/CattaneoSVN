\documentclass[conference]{IEEEtran}
%\usepackage{epsfig}
\usepackage{url}

\usepackage{cite}
\usepackage[dvips]{graphicx}
%\pagenumbering{arabic}  % Arabic page numbers for submission.


\def\disgrace{DISCERN}

%\numberofauthors{3}





\begin{document}
\title{DISCERN: A Collaborative Visualization System for Learning Cryptographic Protocols}



\author{\IEEEauthorblockN{Giuseppe Cattaneo}
\IEEEauthorblockA{Dipartimento di Informatica ed\\ Applicazioni ``R.M. Capocelli'',\\
Universit\`{a} di Salerno,\\
Via Ponte don Melillo,\\ I-84084 Fisciano (SA), Italy\\
Email: cattaneo@dia.unisa.it}
 \and \IEEEauthorblockN{Alfredo De Santis}  \IEEEauthorblockA{Dipartimento di Informatica ed\\ Applicazioni ``R.M. Capocelli'',\\
Universit\`{a} di Salerno,\\
Via Ponte don Melillo,\\ I-84084 Fisciano (SA), Italy\\
Email: ads@dia.unisa.it
}\and
\IEEEauthorblockN{Umberto Ferraro Petrillo}
\IEEEauthorblockA{Dipartimento di Statistica, Probabilit\`a e\\ Statistiche Applicate,\\
Universit\`a di Roma ``La Sapienza'',\\  P.le Aldo Moro 5,\\ 00185 Rome,
Italy.\\
Email: umberto.ferraro@uniroma1.it}}

\maketitle

\begin{abstract}

% Understanding cryptographic protocols may sometimes be difficult due
% to the numerous interactions that may occur between the parties of a
% protocol and because of the complexity of the mathematics it may
% use. In the past, several tools have been proposed to simplify the description of a
% protocol. However, these tools have mainly been intended as
% presentation tools for teachers and provide limited involvement on the part of students.
In this paper we propose a novel approach to the learning of
cryptographic protocols, based on a collaborative role-based visualization
system, \disgrace,\ that helps students to understand a protocol by
actively engaging them in a simulation of its execution.  In
\disgrace,\ each student shares a visual exemplification of a
real-world scenario with other students and impersonates one of the parties involved in
the execution of a protocol. Students may take the role of legal or
malicious parties and are provided with primitives that are useful for
the implementation of several protocols. To achieve a certain security
goal correctly, legal parties have to collaborate and carefully
execute the steps required by the implemented protocol in the correct
order. If any error is made, the security of the protocol is exposed
to the threats coming from other students impersonating malicious
parties. The entire process is run under the supervision of the
teacher.

% Understanding cryptographic protocols may sometimes be difficult due
% to the numerous interactions that may occur between the parties of a
% protocol and because of the complexity of the mathematics it may
% use. Over the past few years, several software visualization tools
% have been proposed to simplify the presentation and description of a
% protocol. However, these tools have mainly been intended as
% presentation tools for teachers and provide limited involvement on the part of students.

% In this paper we propose a novel approach to the learning of
% cryptographic protocols, based on a collaborative visualization system,
% \disgrace\ (Distributed System for Cryptography Education oveR the
% Network), that helps students to understand a  protocol
% by actively engaging them in a simulation of a protocol execution. In
% our system, each student impersonates one of the parties involved in the
% execution of a protocol. Students may take the role of legal or malicious
% parties and are provided with primitives that are useful for the
% implementation of several protocols.

% To achieve a certain security goal correctly, legal parties have to
% collaborate and carefully execute the steps required by the
% implemented protocol in the correct order. If any error is made, the
% security of the protocol is exposed to the threats coming from other
% students impersonating malicious parties. The entire process is run
% under the supervision of the teacher. The resulting system is
% collaborative, because students impersonating legal parties have to to
% coordinate and cooperate with each other to achieve a security
% objective using a cryptographic protocol. It is competitive,
% because students impersonating malicious parties are interested in using all the
% primitives they have and all the errors made by legal parties
% to threaten the security of the protocol being run.
% This approach allows students to see for themselves, and better
% understand, the behavior and (some of) the security properties of a cryptographic protocol by executing one.



% \disgrace\ is implemented in java as a distributed application. Its architecture has been designed to operate efficiently while also guaranteeing each user the responsiveness of a local stand-alone application.
\end{abstract}

% A category with the (minimum) three required fields
% \category{K.3.2}{Computers and Education}{Computer and Information Science Education}

%\terms{Security Experimentation}

% \keywords{Cryptographic protocols, e-learning, active learning,
%   collaborative visualization.}

% \classification{K.3.1 - Computer Uses in Education: Collaborative
%   Learning; K.3.2 - Computer and Information Science Education: Computer science education.}
% \terms{Security.}

\section{Introduction}
\label{se:introduction}

A cryptographic protocol is a distributed algorithm defined by a
sequence of steps indicating precisely the actions required of two or
more entities (also called, parties) to achieve a specific security
objective (\cite{handbook}). The steps of a cryptographic protocol
consist of computational operations or message transmissions that must
be executed in a specific order by the parties involved in the
protocol. The objective of a cryptographic protocol is usually the
transmission of a message under certain security conditions such as
{\em integrity} (ensuring information has not been altered by
unauthorized or unknown means), {\em confidentiality} (keeping
information secret from all but those who are authorized to see it),
{\em authentication} (corroboration of the identity of the entities
involved in the protocol) and {\em message authentication}
(corroborating the source of information).

Understanding cryptographic protocols may sometimes be difficult due
to the numerous interactions that may occur between the parties of a
protocol and because of the complexity of the mathematics behind the
computation operations. For these reasons, cryptographic protocols are
often taught with the help of visual metaphors. The teacher draws a
scenario, typically using electronic slides, where several parties are
engaged in the execution of a cryptographic protocol. The protocol
specification is provided by simulating its execution and visualizing,
at each step, the computation being executed or the message being
transmitted. Moreover, the protocol execution is often simulated using
a real input set, albeit a very simple one, to show the learners the
effects of the mathematical computations used by the protocol.

This approach is very simple to implement and leads to lectures that
tend to be easier to understand than traditional ones. However, it
suffers from the limitations of a passive style of teaching. Several
studies in the past, such as
\cite{chickering1987spg,McConnell96,guthrie2004wdu}, have proved that
involving computer science students in an active learning environment
improves their comprehension and retention of materials.  When
speaking of cryptographic protocols, an interesting form of involvement
for the students would be the possibility of letting them see for
themselves the behavior and (some of) the security properties of a
cryptographic protocol by executing one. To this end, several tools
have been proposed so far to simulate, through visualization, the
execution of a cryptographic protocol. All of these tools share
essentially the same philosophy: they require a person to interact
with an application to dictate/describe all the steps that form a
certain cryptographic protocol: this implies that the student using one of these tools gets, as output, exactly
what he instructs the tool to do.

In this paper, we propose a different approach by making use of the
distributed nature of a cryptographic protocol. In our approach,
several users impersonate different parties involved in the
simultaneous execution of a protocol. They can collaborate in the
execution of the protocol, in order to achieve a specific security
objective, or they can hinder the protocol execution, in order to
disrupt its security. Our expectation is that by properly playing his
part during a protocol execution involving other students and with a
clear objective to be reached under the supervision of the teacher,
a student is better motivated in experimenting with a cryptographic
protocol and is able to reache a stronger comprehension of the protocol
than when using a traditional single-user visualization system,


% We implemented this approach as a java-based collaborative visualization system named
% \disgrace\ (Distributed System for Cryptography Education oveR the Network). A prototype version of this system is available for download.



% Anonymized
We implemented this approach as a java-based distributed system named
\disgrace.\ Our system uses the same visualization and cryptographic
engine as GRACE, an existing visualization tool, and adds all the
complexity needed to handle multi-user protocol execution in a
scalable, consistent, and efficient way. A prototype version of \disgrace\ can be found at:\\ {\tt
 http://www.dia.unisa.it/research/discern/}.


\section{Previous Work}
Over the past few years, several tools have been proposed to support
and simplify the teaching of cryptographic protocols through software
visualization. These tools can produce a visual, sometimes
interactive, description of a protocol, thus replacing traditional
electronic slides. The resulting visualizations typically depict a
scenario where several parties coexist and interact to execute a
protocol. For example, the ProtoViz~\cite{Protoviz} system accepts as
input a protocol description written in a simple specification
language and outputs a visualization of the protocol. The
visualization employs graphical metaphors commonly used in this field,
such as icons displaying a key to represent a cryptographic key, and
is animated in such a way as to represent the stepwise actions of the
protocol.


TECP~\cite{TECP} and GRASP~\cite{Schweitzer06} place stronger emphasis
on the mathematics underlying a protocol, by explicitly visualizing
the computational steps and the information exchanged during a
protocol execution without using animations or graphical
metaphors. Moreover, these tools introduce the concept of {\em
  interactivity} by making it possible for the user to modify on the
fly the specification of a protocol to recreate, for example, what-if
scenarios.

Another interactive visualization system is GRACE~\cite{Cattaneo08},
whose main distinction with respect to other systems is that the
cryptographic primitives it uses are concretely implemented. Running a protocol on real input sets makes it
possible to experience firsthand some of its properties and
weaknesses. The drawback of this system is that a protocol can be
represented only if the cryptographic primitives it uses have been
previously implemented in the system. Finally,
JCrypTool~\cite{jcryptool} is an interactive e-learning application that can be used
to analyze the behavior of several cryptographic algorithms and
protocols. Like GRACE, the cryptographic primitives available in
JCrypTool are concretely implemented and may be run on arbitrary data
sets. On the other hand, this system makes litte use of visual metaphors
to explain cryptographic protocol related concepts and is more focused
on the numerical part of these protocols.


To the best of our knowledge, the collaborative approach has not yet
been tried in the field of software systems for learning cryptographic protocols.
The only collaborative approach we are aware of has instead been proposed in~\cite{Hamey03}
where the author describes a paper-and-pencil game that can be used to simulate
cryptographic protocols and to explore possible attacks against
them. The game is played by several users at a time, each playing a
role in the simulated protocol. By following a provided set of rules
that mimic the cryptographic operations used by a protocol and by
interacting with each other, the learners can recreate a protocol and
deal with some of the key issues in cryptographic protocols, such as
integrity, confidentiality, and non-repudiation in secure data
communications.


\section{Our Contribution}

All the interactive visualization tools presented so far require a
single user to simulate the steps that several parties have to run for the correct
(or incorrect) execution of a protocol. This approach works well with
teachers, i.e., when the focus is only to present a protocol and its properties, and one
clearly has in mind the steps that have to be executed.
This is often not the case when students are interested in learning or improving
their comprehension of a protocol. In such cases, the user has
typically a very limited knowledge of a protocol and its learning
would benefit from the possibility to experience the security
properties of that protocol and to assess the consequences of
errors made during the simulation of the protocol itself.

The process of learning a protocol by experimenting with it may be
difficult to follow if the user is asked to impersonate,
simultaneously, all the parties involved in a protocol execution,
including those threatening the security of the protocol. We thus make
recourse to the original nature of a cryptographic protocol: a
distributed algorithm where several parties have to cooperate to
achieve a security goal. If all the parties behave correctly, the
security goal will be achieved. If one of the parties deviates from
the standard protocol (e.g., by executing two consecutive steps in the
wrong order), the protocol execution may be exposed to security
threats.

Our idea, implemented by a distributed system called \disgrace,\ is to
involve several students simultaneously in the simulation of a
cryptographic protocol, with each user impersonating a distinct party.
All parties share a visual exemplification of a real-world scenario in
which they are connected via a public communication network, can
instantiate and manipulate simple artifacts (e.g., a digital document)
and can perform a set of communication and cryptographic primitives
related to these artifacts. By using these primitives, parties can
recreate several cryptographic protocols and some of their application
scenarios using real input sets. Moreover, these primitives are not
only simulated but concretely implemented: they will behave in the
simulation as they would behave in a real-world implementation.
A key factor is that some of the connected users, even the teacher himself, may behave dishonestly and
may use some ad-hoc primitives (such as eavesdropping on the
communication channel or forging digital credentials) to threaten the
security of the protocol. Thus, the student is engaged in an activity
that may be either {\em collaborative} or {\em competitive}. It is
collaborative because legal (i.e., honest) parties have to to
coordinate and cooperate with each other to achieve a security objective using a cryptographic
protocol. It is competitive because legal parties are interested in
running correctly a protocol in order to prevent malicious (i.e.,
dishonest) parties from accessing a secret information and, on the
other side, malicious parties are interested in using all the
primitives they have and all the errors made by legal parties
to threaten the security of their protocols and access their secret information.

\section{Application Scenarios}
\label{sec:application}
In this section we describe three learning activities, lasting
approximately 10-15 minutes each, where topics related to
cryptographic protocols can be tried out by a small group of students
using \disgrace.\ We will suppose that the
students involved in this activity already have some basic theoretical
knowledge of public-key related cryptographic protocols and introduce
three application scenarios that require the implementation of these
protocols.

For each scenario, we provide a short description of a security
problem to be solved using a cryptographic protocol. We introduce an
initial setting for the session and illustrate a possible protocol
that apparently solves that problem. Moreover, we outline some of the
possible threats to the protocol due to errors performed by legal
parties and/or to actions undertaken by malicious parties. Finally, we
summarize the concepts that we expect students may better understand
through such an experience.

\subsection{ Secure Communication using Public-Key Cryptography}

In a public key setting, each party has a private key that is kept
secret and a public key that is published and accessible to every
other party. The party never publishes or transmits its private key to
anyone. One of the most common applications for public-key
cryptography is ensuring the confidentiality of a transmission. The
sender of a message looks up or is sent the recipient's public key,
and uses it to encrypt the message. The recipient uses his private
key to decrypt the ciphertext received and to obtain the message.
This public key setting allows both secure communication and
digital signatures. One of the most widely used public key cryptographic schemes is RSA~\cite{rivest77}.

{\sc Legal parties:} Alice, Bob

{\sc Malicious parties:} Charlie

{\sc Initial context:} The teacher creates a document $d$ in Alice's
workspace (i.e., the personal area used by parties to store their
documents and cryptographic artifacts) containing the message 'I love you, Bob'. Then, he generates
two distinct pairs of RSA keys, one in Alice's workspace and one in
Bob's workspace. Finally, he asks Alice to deliver document $d$
to Bob using public-key cryptography and asks Charlie to try to
gain access to the content of the message that Alice is about to send to Bob.

{\sc Expected protocol execution:} Alice asks Bob for a copy
of his RSA public key. After obtaining it, Alice uses this key
to encrypt the document $d$ and sends the resulting text to Bob, who decrypts it by using his private key.

{\sc Possible problems:} 1. Bob publishes his private key instead of
his public key. 2. Alice uses her public key or her private key,
instead of Bob's public key, to encrypt the document to send to Bob.
3. Charlie forges a pair of cryptographic keys impersonating Bob and
makes the resulting key available to Alice. Alice uses Charlie's
public key to encrypt the document and sends it to Bob. Charlie
intercepts the document and decrypts it using his private key.

{\sc Learning task:} The primary task is to understand how
a public-key communication scheme works and the role played by
cryptographic keys in these schemes. Moreover, students may also
experience the problem of establishing the real identity
of the users participating in a communication scheme and the possible
security threats due to the usage of keys forged by malicious users.


\subsection{ Using Public-key Cryptography in a Voting Scheme}

A possible application for public-key cryptography is the
implementation of electronic voting schemes. A group of
parties ${\mathcal G}$ may be requested to express
a preference among a set of possible choices. The choice
should be transmitted electronically to a trusted authority
in such a way as to guarantee the confidentiality of the vote.


A simple voting scheme can be implemented by having each party
write his preference in an electronic document to be encrypted
using the trusted authority public key. The encrypted document
is then sent to the trusted authority which, in turn, uses
its private key to decrypt it. The scheme can be further
improved by using digital signatures to verify the identity of the parties that sent in their preferences.

{\sc Legal parties:} Alice, Bob, Charlie, Dave

{\sc Malicious parties:} Eve

{\sc Initial context:} The teacher creates a pair of RSA keys
in Dave's workspace. Then, he reveals to all parties the set of
possible choices to cast votes on. Finally, he asks Dave to
impersonate the trusted authority and asks Alice, Bob,
and Charlie to cast their preferences to Dave, and asks Eve to determine which preference has been expressed by each party.

{\sc Expected protocol execution:} Alice asks Dave for
a copy of his public RSA key and uses it to encrypt the
document containing her choice. The resulting text is sent to Dave, who decrypts it by using his private key.

{\sc Possible problems:} Eve writes all the possible choices in
different documents in her workspace and encrypts these documents
using Dave's public key. When one of the legal parties transmits
his preference to Dave, Eve intercepts the encrypted document
and compares its content to the content of all of her documents.
If a match is found, the preference expressed by the intercepted party is revealed.

{\sc Learning task:} The primary task is to demonstrate that
when the universe of possible messages to encrypt is small,
public-key cryptography may be incapable of providing an adequate
level of security. In the discussed case, the problem can easily
be solved by the voters inserting into the document
containing their preferences, additional text
(e.g., some random characters) whose content cannot be predicted by malicious parties.

\begin{figure*}
\centering
\includegraphics[width=12cm]{scenario1}
\caption{A screenshot of \disgrace\ at the beginning of a new session, as seen by Alice. Bob is a legal party. Alice is impersonating a malicious party (recognizable by the ear-shaped icon beside her portrait). The teacher is impersonating himself (recognizable by the crown-shaped icon).}
\label{fig:scenario}
\end{figure*}

\subsection{ Trusting Public Keys and Certificates}
In order for a public-key secure communication scheme to work,
the sender of a message needs the public key of the recipient.
Since it is not always possible to assume that this information
is available {\em a priori}, there is the need for a mechanism
that would allow the recipient to publish his public credentials, and
for the sender to acquire them.

This mechanism can be implemented using Certificate Authorities (CA). These
are third parties, trusted by all the participants in a protocol, whose
role is to provide digital certificates attesting the correct
identities of all the parties involved in a protocol. A digital
certificate includes the public key of the party it has been issued
to. Another party may verify the integrity and the authenticity of a
digital certificate by using the CA public key (which is supposed to
be known a priori to all the participants in a protocol). One of the
most widely used CA implementation is the one based
on X.509 digital certificates \cite{x509}.

{\sc Legal parties:} Alice, Bob, Charlie

{\sc Malicious parties:} Dave

{\sc Initial context:} First, the teacher creates a document $d$
in Alice's workspace containing the text 'I love you, Bob'.
Then, he creates a new certification authority (CA) and uses it to issue
two digital certificates: one for Alice and one for Bob. Next, the
teacher sends to each participant his digital certificate and the key
of the CA.  Subsequently, he reveals
to Dave that Alice will send Bob a document that Dave has to
intercept and read. Lastly, he instructs Alice to deliver
document $d$ to Bob using public-key cryptography. {\em Bob receives
istructions to do nothing}.

{\sc Expected protocol execution:} Dave forges a new digital
certificate which is apparently entitled to Bob and sends it to
Alice. If Alice chooses to test the authenticity of the received
certificate by verifying it using the CA public key, she will realize
that the certificate has been forged and will stop executing the
protocol. Otherwise, Alice will use the key included in the forged
Bob's certificate to encrypt message $d$: the outcoming
encrypted message will be sent to Bob. Dave will eavesdrop on the
communication channel, acquire a copy of the exchanged message
and use the private key he owns to access the message content.

{\sc Learning task:} The learning task proves the need
for a mechanism to establish the identity of the parties
involved in a public-key secure communication scheme.
Without this mechanism, it is relatively easy for
a malicious party to forge credentials to appear as if
they were generated by the recipient of a transmission
and to decipher messages encrypted using these credentials.

\begin{figure*}
\centering
\includegraphics[width=12cm]{permissions}
%\epsfig{file=permissions.eps, height=5cm, width=8cm}
\caption{The control panel the teacher uses to manage the interaction
  capabilities of all the other parties executing a protocol. In the
  setup phase the only active party is the teacher himself (i.e.,
  Rivest, in this case).}
\label{fig:permissions}
\end{figure*}


\section{DISCERN}
\label{sec:dgrace}
Our system has been implemented in Java as a multi-user distributed
application. The simulation of cryptographic protocols takes place in
learning sessions where each session is organized in two phases:
{\em setup} and {\em operational}. In the setup phase, all users
(including the teacher) connect to the \disgrace\ server using
a provided application and by choosing a name, a picture, and a
role for the party they will impersonate.  Users joining a new
session are presented with a graphical window visualizing
the simulated scenario and all the parties that are connected so far.
Each party has a {\em workspace} that contains all of his artifacts
(e.g., cryptographic keys, plain and encrypted documents, digital
certificates) accompanied by his name and portrait.
Workspaces are split into two areas: a {\em public area},
colored green, whose content is visible to all users and
a {\em private area}, colored red, whose content is only
visible to its owner and to the teacher. (see Figure \ref{fig:scenario} for an example).

During this phase, the teacher configures the initial environment
for the protocol execution. He can choose which cryptographic
primitives will be available during the execution and can manipulate
directly the contents of each party's workspace. Once the initial
environment is ready, the teacher instructs the parties regarding their goals and advances the session to the following phase.

In the operational phase, the connected parties are free to use their
primitives and interact according to the instructions given by the
teacher. Users' interactions can be disabled at any time by the
teacher, who may temporarily take exclusive control of the session in
order to modify the current environment or comment on the current
state of execution of the protocol.

Users may join a session by playing one of three different roles:

\begin{itemize}
\item{\bf Legal.} It is a party interested in running a cryptographic protocol
to share some secret information with other parties under certain
security conditions.

\item{\bf Malicious.} It is a party interested in
disrupting the security of the protocols implemented by legal parties.
In addition to the primitives assigned to legal parties, malicious
parties also have the possibility of capturing a copy of all the
artifacts exchanged by other parties and of forging artifacts that
will appear as generated by any of the other parties.

\item{\bf Teacher.} This role includes all of the primitives of the malicious and
legal parties plus the possibility of ending a session, disabling or
re-enabling the interaction capabilities of any of the other parties,
or temporarily taking control of the party impersonated by another
user. In addition, the teacher can send private or public text
messages to the other parties of a session and can choose to add two
additional entities to a session: a {\em public storage area}, to share
cryptographic keys, and a {\em digital certification authority}, to
issue and verify digital certificates. A screenshot of the control
panel the teacher uses to manage permissions for the other parties is shown in
Figure \ref{fig:permissions}.
\end{itemize}

During a session, the only role that is publicly known is that of the
teacher. Moreover, the visualizations of all the actions that may
reveal the identity of a malicious party (i.e., eavesdropping on a
communication channel) are only visible to that party and to the
teacher.


\subsection{The User Interface}
% anonimyzed

The user interface of \disgrace\ is an evolution of the one used in GRACE and is based on the concepts of {\em
  parties} and {\em artifacts}. A party is an entity that may actively participate in a protocol and is
impersonated by a user. As already mentioned in Section
\ref{sec:dgrace}, he is visualized through his portrait and his
workspace. The artifacts are the digital information needed for the
execution of a cryptographic protocol (e.g., an encrypted document, a
cryptographic key, a digital signature) and are represented using the
visual metaphors traditionally used for these concepts, as shown in
Figure \ref{fig:workspace}. Each artifact may exist in multiple copies
and may be found in the public or the private area of one or more
workspaces.  A party can make his copy of an artifact public by dragging the icon
representing it over the green area, and vice-versa. Moreover, each
artifact has a set of information describing its properties that can
be accessed by right-clicking on it. This information is usually
established during the artifact's creation and includes the identity
of the party that created that artifact. Malicious users can forge
this information and make the artifact appear as if it was generated
by any other of the parties connected to the same session.  Parties
and artifacts have the following set of associated tasks:

\begin{figure*}
\centering
\includegraphics[width=12cm]{workspace1}
%\epsfig{file=workspace.eps, height=6cm, width=8cm}
\caption{The workspace of Alice. Her private area contains her RSA
  private key, Bob's RSA public key and an unencrypted document. Her
  public area contains her RSA public key and an encrypted copy of her
  document. She is now engaged in forging a new pair of cryptographic
  keys and, therefore, she is choosing the identity of the party to
  impersonate.}
\label{fig:workspace}
\end{figure*}

\begin{itemize}
\item{\bf Cryptography related tasks.} Cryptography related artifacts
  may have cryptographic algorithms associated to them. For instance,
  a cryptographic key may come with two algorithms: one for encrypting
  a text and another for decrypting a text. These algorithms may be
  triggered by combining the icon representing a cryptography-related
  artifact with another artifact. By doing so, the system will pop-up
  a list of all the cryptographic algorithms implemented by that
  artifact which are compatible with the destination artifact. After
  choosing one of them, the system will run the algorithm and will
  show the mathematical elaborations performed by that algorithm in a
  text window on all the clients.

\item{\bf Communication related tasks.} Parties involved in a protocol
  execution can exchange artifacts by offering them or by taking
  them. In the first case, a user drags an artifact from the workspace
  of his party to the workspace of another party: a confirmation
  window will pop-up on the destination client, asking the other user
  if he accepts that artifact. If he agrees, a copy of the artifact is
  transferred to the private area of the destination party. A user can
  also drag one of the public artifacts of a destination party over to
  his workspace: in this case, the transmission takes place
  immediately, without any further confirmation. In both cases, all
  the malicious parties participating in the scenario will
  automatically acquire a copy of the artifacts being exchanged.
\end{itemize}

A user can play his part in a cryptographic protocol by using his
party to instantiate the proper artifacts and run the cryptography and
communication-related tasks needed for the correct execution of the
protocol in a timely manner.




\subsection{Implementation Notes}
Basically, \disgrace\ has been designed considering two opposite
needs. On the one hand, we had to guarantee that the content of a
visualization be identical to all users participating in the same
session and that simultaneous users' interactions would be processed
in such a way as to guarantee the coherency of the simulation. On the
other hand, we were interested in designing a system capable of
operating efficiently and without particular delays in user interface
responsiveness, even when several users were connected at the same
time.


These requirements have been fulfilled by designing an architecture
where all the users participating in the same session are connected to
a centralized server and maintain, in their client application, a
local copy of the session. The role of the server is to coordinate and
synchronize the protocol execution. On the client side, all the
parties and artifacts existing in a session are modelled as objects,
and all the actions they support are implemented as methods. These
objects exist at two levels. At an application level, they provide a
concrete implementation for the concepts they describe (e.g., the
object {\tt Document} holds a text document) and for the actions they
support. At a visualization level, they are represented using visual
metaphors.

Each user is free to interact with his own local copy of the session,
browse the workspace of other parties or read the properties of his
artifacts without the need to communicate with the remote
server. Whenever one of the users asks for an action that has the
effect of modifying the content of the session (e.g., creating a new
document), the corresponding request is sent to the server. Once
received, the request is initially put in a queue and then eventually
processed if the requesting user has the permission to perform that
action. When committed, a request is sent to the clients of all users
participating in the session, where it will be run both at an
application and at a visualization level.

This approach produced several important advantages. Firstly, it is
efficient, because it allows a visualization to be updated by just
communicating the details of the action to be executed, usually in a
few hundred bytes, and not the payload required to describe graphic
context changes. Secondly, it guarantees the coherency of the
visualization as actions cannot be executed simultaneously. Thirdly,
it simplifies the implementation of the different roles supported in
\disgrace,\ as the server trivially decides which request to forward
according to the identity of the requester and to his role.

Notice that this solution does not prevent the possibility for users
of issuing interaction requests that will not be executed because of
actions performed by other users (e.g., a user asks another user for a
copy of an artifact that has been, in the meantime, deleted). In these
cases, the server side will be unaware of these problems and will
still commit these requests to the connected clients that will
silently discard them.

% Anonymized

% With regards to the cryptographic issue, this is an evolution of the
% one already used in the GRACE system (see ~\cite{Cattaneo08}). A
% standard set of cryptographic primitives with associated
% visualizations has been defined in \disgrace.\ This set includes
% several of the cryptographic algorithms commonly taught in security
% courses. The user interested in providing his own algorithms has to
% code them as java classes compliant with the corresponding \disgrace\
% interfaces.

% The cryptographic part has been implemented in \disgrace\ by
% introducing a set of cryptographic primitives with associated
% visualizations.

With regards to the cryptographic issue, this is an evolution of the
one already used in the GRACE system (see ~\cite{Cattaneo08}). A
standard set of cryptographic primitives with associated
visualizations has been defined in \disgrace.\ This set includes
several of the cryptographic algorithms commonly taught in security
courses and takes advantages of the implementations available
in the Java Cryptographic Extension \cite{jce} standard package.

The user interested in providing support for other algorithm or
cryptographic protocols has to code them as java classes compliant
with the corresponding \disgrace\ interfaces.


\section{Conclusions and Open Issues}
In this paper we presented \disgrace,\ a system to help students learn
how to use cryptographic protocols. \disgrace\ actively engages
students in a simulation of a cryptographic protocol, where each
student impersonates one of the parties executing the protocol.
Our system proposes an approach to the learning of a cryptographic
protocol that is collaborative and competitive. It is collaborative,
because students impersonating honest parties have to cooperate
and coordinate themselves in order to correctly run a protocol.
It is competitive because students impersonating dishonest parties
will use their primitives to deceive honest parties and take
advantages of their errors in order to threaten the security of the
protocol being run. Students may collectively experience some of the
consequences of these errors and the way they disrupt the security
properties of a protocol, because the cryptographic primitives
available with \disgrace\ and used in a protocol execution are not
only simulated but concretely implemented.

There are still some significant issues that need to be
investigated in our work. First of all, there is need of a deeper study on
the effects that this approach may have on
the students' understanding of these topics. In our case, we performed
several tests with students participating to a Security on
Communication Networks undergraduated course during the fall 2008
semester and at debugging and testing the system. In these tests we
organized our students in groups, with each group featuring a
malicious party whose role was unknown to ther other parties, and
asked each group to recreate one of the two application scenarios
presented in Section \ref{sec:application} and based on the usage of public-key cryptography.
During these tests, we observed a positive effect on the students
about their understanding of cryptographic protocols, and we
experienced that the competitive factor was the main point driving the
attention and the actions of the students. However, we are aware there is the need
for a more extensive and thorough investigation that should compare
our approach with both the approaches based on the usage of
electronic-slides and on the usage of non-collaborative cryptographic
protocols learning tools.

Another significant issue arisen during our experimentation concerns the efforts
to be spent for preparing a learning session versus the number of
students thay may participate in it. On a side, the teacher may have
to spend a not-so-small amount of time to setup complex learning
sessions. On the other side, the number of students that may be
actively involved in a protocol execution at a time is relatively
small. So, the implementation of this approach seems to be problematic when dealing
with large classes and requires the development of solutions for
scaling on the number of students participating in a session without
impacting negatively on their learning experience.

There are several possible directions to be investigated to this
end. A possible solution would be to make it possible for a teacher to
initiate and manage more than one learning session at a same
time. Another interesting direction is to allow many students to join a learning
session as viewers, without a party to impersonate initially, but with
the possibility for the teacher to assign them a party to play at any
time during the execution of a protocol.

\bibliographystyle{IEEEtran}
\bibliography{IEEEabrv,discern_collaboratecom}

\end{document}
