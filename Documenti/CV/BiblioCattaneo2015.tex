% !TEX encoding = UTF-8 Unicode


\documentclass[11pt,a4paper,sans]{moderncv}        % possible options include font size ('10pt', '11pt' and '12pt'), paper size ('a4paper', 'letterpaper', 'a5paper', 'legalpaper', 'executivepaper' and 'landscape') and font family ('sans' and 'roman')

% moderncv themes
\moderncvstyle{classic}                            % style options are 'casual' (default), 'classic', 'oldstyle' and 'banking'
\moderncvcolor{blue}                               % color options 'blue' (default), 'orange', 'green', 'red', 'purple', 'grey' and 'black'
%\renewcommand{\familydefault}{\sfdefault}         % to set the default font; use '\sfdefault' for the default sans serif font, '\rmdefault' for the default roman one, or any tex font name
% \nopagenumbers{}                                 % uncomment to suppress automatic page numbering for CVs longer than one page

% adjust the page margins
\usepackage[scale=0.8, top=2.5cm, bottom=2.5cm]{geometry}
\recomputelengths 
%\setlength{\hintscolumnwidth}{3cm}                % if you want to change the width of the column with the dates
%\setlength{\makecvtitlenamewidth}{10cm}       % for the 'classic' style, if you want to force the width allocated to your name and avoid line breaks. be careful though, the length is normally calculated to avoid any overlap with your personal info; use this at your own typographical risks...

\usepackage[T1]{fontenc}
% character encoding
\usepackage[utf8]{inputenc}                       % if you are not using xelatex ou lualatex, replace by the encoding you are using
\usepackage[english,italian]{babel}

% bibliography with mutiple entries
%\usepackage{multibib}
%\newcites{book,misc}{{Books},{Others}}

\usepackage[style=numeric-comp,sorting=ydnt,defernumbers]{biblatex}
\addbibresource{personal.bib}

\AtDataInput{%
  \csnumgdef{entrycount:\strfield{prefixnumber}}{%
    \csuse{entrycount:\strfield{prefixnumber}}+1}}

\DeclareFieldFormat{labelnumber}{\mkbibdesc{#1}}    
\newrobustcmd*{\mkbibdesc}[1]{%
  \number\numexpr\csuse{entrycount:\strfield{prefixnumber}}+1-#1\relax}


% to show numerical labels in the bibliography (default is to show no labels); only useful if you make citations in your resume
\makeatletter
\renewcommand*{\bibliographyitemlabel}{\@biblabel{\arabic{enumiv}}}
\makeatother
\renewcommand*{\bibliographyitemlabel}{[\arabic{enumiv}]}% CONSIDER REPLACING THE ABOVE BY THIS


\usepackage{lastpage}
\usepackage[official]{eurosym}

\newcommand{\lastYear}{\em 2015} 

% personal data
\name{Giuseppe}{Cattaneo}
\firstname{Giuseppe}
\familyname{Cattaneo}

\title{Curriculum Vit\ae{}}                               % optional, remove / comment the line if not wanted
\address{Via Panoramica, 15}{I-84134 Salerno}{Italy}% optional, remove / comment the line if not wanted; the "postcode city" and "country" arguments can be omitted or provided empty
\phone[mobile]{+39~320~7406~160}         % optional, remove / comment the line if not wanted; the optional "type" of the phone can be "mobile" (default), "fixed" or "fax"
\phone[fixed]{+39~089~96~9716}
\phone[fax]{+39~089~96~9840}
\email{cattaneo@unisa.it}                          % optional, remove / comment the line if not wanted
%\homepage{www.dia.unisa.it/professori/cattaneo}                         % optional, remove / comment the line if not wanted
%\social[linkedin]{john.doe}                        % optional, remove / comment the line if not wanted
%\social[twitter]{jdoe}                                  % optional, remove / comment the line if not wanted
%\social[github]{jdoe}                                  % optional, remove / comment the line if not wanted
%\extrainfo{additional information}              % optional, remove / comment the line if not wanted
\photo[64pt][0.4pt]{FotoTessera2010.png}   % optional, remove / comment the line if not wanted; '64pt' is the height the picture must be resized to, 0.4pt is the thickness of the frame around it (put it to 0pt for no frame) and 'picture' is the name of the picture file
%\quote{Some quote}                                  % optional, remove / comment the line if not wanted

\makeatletter
\renewcommand*{\nopagenumbers}{\@displaypagenumbersfalse}
\AtEndPreamble{%
  \AtBeginDocument{%
    \if@displaypagenumbers%
      \@ifundefined{r@lastpage}{}{%
        \ifthenelse{\pageref{lastpage}>1}{%
          \settowidth{\pagenumberwidth}{\color{color2}\addressfont\itshape\strut\thepage/\pageref{lastpage}}%
          \fancypagestyle{plain}{%
            \fancyfoot[r]{\parbox[t]{\pagenumberwidth}{\color{color2}\addressfont\itshape\strut\thepage/\pageref{lastpage}}}}% the parbox is required to ensure alignment with a possible center footer (e.g., as in the casual style)
          \pagestyle{plain}}{}}%
      \AtEndDocument{\label{lastpage}}\else\fi}}
\makeatother

\cfoot{\addressfont\itshape\textcolor{color2}{Pagina \thepage\ / \pageref{LastPage}}}

\renewcommand{\headrulewidth}{0.4pt}

\makeatletter
\chead{\addressfont\itshape\textcolor{color2}{Curriculum Vit\ae{} di \@firstname~\@familyname}}
\makeatother

%----------------------------------------------------------------------------------
%            content
%----------------------------------------------------------------------------------
\begin{document}

\thispagestyle{empty}


\nocite{*}

\renewcommand\refname{Bibliografia}
\printbibheading
\printbibliography[prefixnumbers={A},type=article,title={Articoli su riviste internazionali},heading=subbibliography]
\printbibliography[prefixnumbers={C},type=inproceedings,title={Articoli a conferenze},heading=subbibliography]

\clearpage
%-----       letter       ---------------------------------------------------------
% recipient data
%\recipient{Company Recruitment team}{Company, Inc.\\123 somestreet\\some city}
%\date{January 01, 1984}
%\opening{Dear Sir or Madam,}
%\closing{Yours faithfully,}
%\enclosure[Attached]{curriculum vit\ae{}}          % use an optional argument to use a string other than "Enclosure", or redefine \enclname
%\makelettertitle
%
%Lorem ipsum dolor sit amet, consectetur adipiscing elit. Duis ullamcorper neque sit amet lectus facilisis sed luctus nisl iaculis. Vivamus at neque arcu, sed tempor quam. Curabitur pharetra tincidunt tincidunt. Morbi volutpat feugiat mauris, quis tempor neque vehicula volutpat. Duis tristique justo vel massa fermentum accumsan. Mauris ante elit, feugiat vestibulum tempor eget, eleifend ac ipsum. Donec scelerisque lobortis ipsum eu vestibulum. Pellentesque vel massa at felis accumsan rhoncus.
%
%Suspendisse commodo, massa eu congue tincidunt, elit mauris pellentesque orci, cursus tempor odio nisl euismod augue. Aliquam adipiscing nibh ut odio sodales et pulvinar tortor laoreet. Mauris a accumsan ligula. Class aptent taciti sociosqu ad litora torquent per conubia nostra, per inceptos himenaeos. Suspendisse vulputate sem vehicula ipsum varius nec tempus dui dapibus. Phasellus et est urna, ut auctor erat. Sed tincidunt odio id odio aliquam mattis. Donec sapien nulla, feugiat eget adipiscing sit amet, lacinia ut dolor. Phasellus tincidunt, leo a fringilla consectetur, felis diam aliquam urna, vitae aliquet lectus orci nec velit. Vivamus dapibus varius blandit.
%
%Duis sit amet magna ante, at sodales diam. Aenean consectetur porta risus et sagittis. Ut interdum, enim varius pellentesque tincidunt, magna libero sodales tortor, ut fermentum nunc metus a ante. Vivamus odio leo, tincidunt eu luctus ut, sollicitudin sit amet metus. Nunc sed orci lectus. Ut sodales magna sed velit volutpat sit amet pulvinar diam venenatis.
%
%Albert Einstein discovered that $e=mc^2$ in 1905.
%
%\[ e=\lim_{n \to \infty} \left(1+\frac{1}{n}\right)^n \]
%
%\makeletterclosing

%\clearpage\end{CJK*}                              % if you are typesetting your resume in Chinese using CJK; the \clearpage is required for fancyhdr to work correctly with CJK, though it kills the page numbering by making \lastpage undefined
\end{document}


%% end of file `template.tex'.
