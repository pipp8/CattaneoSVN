% !TEX encoding = UTF-8
% !TEX program = pdflatex
% !TEX spellcheck = it_IT

\documentclass[italian,a4paper]{europasscv}
\usepackage[italian]{babel}

\usepackage{tabularx}

\usepackage{lastpage}

\usepackage[official]{eurosym}

%% DATES, VERSIONS AND TITLES:
\usepackage{svn-multi}

\svnid{$Id: CVCattaneoEuropass.it.tex 917 2024-03-28 21:27:26Z cattaneo $}

\svnkwsave{$LastChangedDate $}

\newcommand{\currentDate}{\svnfileday/\svnfilemonth/\svnfileyear}
\newcommand{\lastYear}{\em 2024} 

% bibliografia 
%\usepackage[style=numeric-comp,sorting=ydnt,defernumbers]{biblatex}
%\addbibresource{personal.bib}

\usepackage[backend=biber,style=numeric-comp,giveninits=true, maxbibnames=10,sorting=ydnt,defernumbers=true,doi=false,isbn=false,url=false]{biblatex}
%\usepackage[defernumbers=true,backend=biber]{biblatex}

\addbibresource{BiblioCattaneo.bib}

\usepackage{europasscv-bibliography}


\DeclareRefcontext{rc}{sorting=ydnt}

\AtEveryBibitem{%
  % elimina i seguenti field dalla bibliografia
  % \clearfield{note}%
  \clearlist{language}%
}

\DeclareSourcemap{
  \maps[datatype=bibtex]{
    \map{
       \step[fieldset=note, null]
       \step[fieldset=eprint, null]
    }
  }
}
%\AtDataInput{%
%  \csnumgdef{entrycount:\strfield{prefixnumber}}{%
%    \csuse{entrycount:\strfield{prefixnumber}}+1}}
%
%\DeclareFieldFormat{labelnumber}{\mkbibdesc{#1}}    
%\newrobustcmd*{\mkbibdesc}[1]{%
%  \number\numexpr\csuse{entrycount:\strfield{prefixnumber}}+1-#1\relax}


\ecvname{Giuseppe Cattaneo}
\ecvaddress{Via Panoramica, 15, 84135 Salerno, Italia}
\ecvmobile{(+39) 320 7406160}
\ecvtelephone{(+39) 089 96 9716}
\ecvworkphone{(+39) 089 96 9840}
\ecvemail{cattaneo@unisa.it}
\ecvhomepage{https://docenti.unisa.it/000763/home}

\ecvdateofbirth{11 Gennaio 1960}
\ecvnationality{Italiana}
\ecvgender{Maschio}

\ecvpicture[width=3.8cm]{FotoTessera2016.jpg}
%\ecvpictureright

\begin{document}
  \begin{europasscv}

  \ecvpersonalinfo

  \ecvbigitem{Motivo}{Partecipazione al convegno:E-Health: Innovazione e Sviluppo – Interventi per migliori infrastrutture e contenimento della migrazione sanitaria nel Mezzogiorno”}

  \ecvsection{Esperienza Lavorativa}
  
   \ecvtitle{1984 -- \lastYear}{Professore universitario}
%    \ecvitem{}{British Council\newline 123, Bd Ney, 75023 Paris (France)}
%    \ecvitem{}{Evaluation of European Commission youth training support measures for youth national agencies and young people}
% 
	\ecvitem{1999 -- \lastYear}{Professore Universitario di IIa fascia presso Università di Salerno\newline
	Partecipando al concorso bandito dall’Università di Salerno per posti di professore associato seconda fascia settore K05B (Informatica, oggi INF/01), consegue l’idoneità nella procedura comparativa e prende servizio presso la Facoltà di Scienze MM.FF. e NN. nel corso dell’anno accademico 1999-2000 divenendo titolare dell’insegnamento di Linguaggi di Programmazione II fino al 2005. Contemporaneamente ha attivato un nuovo corso denominato Programmazione su reti II, per l’insegnamento delle tecnologie emergenti nell'ambito dei sistemi distribuiti, quali Java Enterprise Edition e ambienti di programmazione evoluti per la creazione di servizi WEB based, secondo il paradigma delle Service Oriented Architecture.\newline
	Successivamente, all’introduzione dei nuovi ordinamenti didattici, ha insegnato Sistemi Operativi alla laurea triennale e Sistemi Operativi Avanzati alla laurea Magistrale.\newline
	Nel 2018 ha conseguito l'Abilitazione  Scientifica Nazionale (art. 16, comma 1, Legge 240/10) per professore di Ia fascia per i settori concorsuali 01/B1 Informatica (INF/01) e 09/H1 Ingegneria Informatica - Sistemi di Elaborazione delle Informazioni (ING-INF/05).}
	
	\ecvitem{1986--1999}{Ricercatore Universitario presso Università di Salerno\newline
	Vincitore di un concorso per ricercatore universitario gruppo di discipline 92/bis (Informatica, oggi INF/01) prende servizio presso la Facoltà di Scienze Matematiche Fisiche e Naturali dell’Università di Salerno, afferendo al Dipartimento di Informatica ed Applicazioni (DIA).}
	 
	\ecvitem{1986--1990}{Invited researcher presso il Laboratoire d'Informatique Théorique et Programmation (LITP), Universitè Paris 6, Parigi, Francia\newline
	Nell'ambito di una collaborazione scientifica tra l’Università di Parigi 6 ed il DIA si reca in visita presso il laboratorio francese (LITP) grazie ad una borsa di studio annuale del FORMEZ.\newline
	Al termine della borsa, il soggiorno è stato prolungato per ulteriori 24 mesi grazie ad un contratto di ricerca biennale (n. 871B00-7909245-LAAISLCIMAIA) stipulato con il LITP dedicato allo sviluppo, la realizzazione e la messa a punto di una Lisp Machine (denominata MAIA) interamente progettata e costruita in Francia nell’ambito di un progetto del Centre National d'Etudes en Télécommunications (CNET).}
 
	\ecvitem{1984--1986}{Tecnico Laureato presso il Dipartimento di Informatica ed Applicazioni dell'Università di Salerno\newline Stipula un contratto triennale ai sensi dell'art. 26 D.P.R. 380/80 per attività specialistica di supporto alla attivazione dei laboratori didattici del Dipartimento in fase di creazione.}
  
  \newpage
  \ecvtitle{}{Attività per il trasferimento tecnologico}
  
  	\ecvitem{1998--2004}{Responsabile R\&D Sintel S.p.A.\newline
	Stipula un contratto con la Sintel S.p.A. per assumere l’incarico di responsabile della divisione ricerca e sviluppo. Nello stesso anno partecipa al collocamento in borsa del titolo della capogruppo (Finmatica S.p.A.) contribuendo alla definizione delle strategie per l'innovazione, parametri cruciali per un'azienda che all'epoca ha caratterizzato il fenomeno della new economy. Per garantire il necessario impegno nello svolgimento di queste attività è stata scelta l'opzione del tempo definito come regime di impegno con l'Università di Salerno attivando un ciclo virtuoso di sinergie tra Università e mondo privato.\newline
  	Dal termine dell'OPA fino al giugno del 2002 ha fatto parte del {"\em Board of Directors"} di Finmatica con l’incarico di  responsabile dei processi per l'innovazione tecnologica all’interno dell’intero gruppo.\newline
  	Nel contempo in collaborazione con la divisione M\&A ha partecipato a tutte le {\em “Due Diligence”} effettuate nel processo di crescita del gruppo mediante acquisizione di nuove aziende affermate su mercati internazionali complementari. Tali acquisizioni si sono dirette prima verso il mercato della sicurezza informatica (Intesis S.p.A.) e successivamente verso l’Extended SCM (Ortéms s.a., Lyon Fr e Mercia Ltd Birmingham UK). Infine particolare interesse è stato rivolto verso il settore della logistica e dei trasporti 3PL con l’acquisizione di OBSoft Paris Fr ed un grande progetto per Deutch Post (DHL).}
  
  	\ecvitem{2013--2018}{Fondatore dello spin-off universitario eTuitus s.r.l. con Infocert\newline
  	Grazie al supporto del partner incubatore Infocert S.p.A. lo spin-off ha potuto travasare competenze, idee innovative e soluzioni applicative nel contesto dei servizi offerti dalla società partner.
  	In particolare sono state progettate soluzioni attualmente ancora in esercizio, come ad esempio il sistema di autenticazione federato SPID, che al momento gestisce circa 5 Milioni di identità digitali.
  	Analogamente, lo spin-off ha proposto e realizzato soluzioni innovative nell'ambito dello storage remoto sicuro (dati cifrati in maniera trasparente all'utente) valutate sia in termini di sicurezza che di usabilità.\newline
  	La società è tuttora in attività ed impiega stabilmente circa 15 specialisti di sicurezza IT. Dal 2018 il sottoscritto è uscito dalla compagine societaria.
  	}
  
	\ecvitem{2017--\lastYear}{Socio fondatore dello spin-off dell'Università di Salerno denominato UniDoc s.r.l.\newline 	
	Nel giugno del 2017, il sottoscritto, unico rappresentante dell'Università di Salerno, con il 51 \% del capitale sociale ha avviato un progetto imprenditoriale in collaborazione con il socio incubatore Consorzio CSA.\newline 
	In sostanza il piano industriale prevedeva l'offerta alla PA ed in particolare al mondo della Sanità Pubblica di servizi e soluzioni atte a promuovere e valorizzare gli sforzi per la digitalizzazione dei processi interni. Più nello specifico, a partire dall'enorme quantità di documenti cartacei raccolti e digitalizzati negli anni dal partner il team di professionisti si è dedicato alla estrazione della conoscenza per alimentare soluzioni basate sull'Artifical Intelligence. L'iniziativa di grande successo è stata ulteriormente spinta dagli effetti della pandemia (che tipicamente hanno spinto a sostituire attività in presenza con servizi on-line) e dalle misure del PNRR che hanno promosso e finanziato le PA verso la digital transformation. Fattore chiave del successo è stato però identificato nella capacità del gruppo imprenditoriale di individuare percorsi di migrazione al digitale che hanno sempre tenuto in considerazione tutta la preesistenza, come è avvenuto ad esempio per le cartelle cliniche che si stanno gradatamente trasformando in cartelle cliniche elettroniche, senza mai perdere il contatto con gli ultimi 20 anni di storia clinica del paziente, contenuti in cartelle cartacee.
	Infine, grazie a soluzioni basate sull'AI è stato possibile, oltre all'estrazione dei dati, ridisegnare i processi interni delle PA per evitare di continuare a produrre o a utilizzare supporti cartacei. Esempio evidente è rappresentato dal progetto Digital Libraries che tende ad aumentare la fruibilità (oltre che a preservare ) dell'enorme patrimonio di beni culturali disponibile nel nostro Paese trasformandolo in formati digitali con opportuni processi di dematerializzazione e metadatazione.  
	Nel 2023 UniDoc ha terminato il percorso di start-up ed è diventata PMI Innovativa, con un fattirato lordo di circa 1 milione € e circa 10 dipendenti a tempo pieno.
	}
	\newpage
%  \ecvblueitem{Organisational / managerial skills}{
%  \begin{ecvitemize}
%    \item whilst working for a Brussels based refugee NGO ``Convivial'' I organized a ``Civil Dialogue'' between refugees and civil servants at the European Commission 20th June 2002
%    \item during my PhD I organised a seminar series on research methods
%  \end{ecvitemize}
%  }
\ecvtitle{}{Attività di coordinamento}{
%\subsection{TBD}
	\ecvitem{1991--\lastYear}{Delegato del Rettore per il polo GARR dell’Ateneo Salernitano\newline
	Fin dall'assunzione ha collaborato alla gestione del centro di calcolo del DIA programmando la migrazione dai sistemi legacy (Digital VMS) ai sistemi Open (Unix 4.2 BSD).\newline
	Dall'ottobre del 1991, con l'entrata in esercizio della rete GARR (Gruppo Armonizzazione Reti Ricerca) voluta dal Ministero dell'Università e della Ricerca Scientifica e Tecnologica (MURST), ha coordinato la nascita del nodo dell'Ateneo Salernitano, nel ruolo di responsabile del Polo GARR, delegato del Rettore.}

	\ecvitem{2012--2015}{Membro del CTS del GARR\newline
	Per il triennio 2012--2015 è stato membro effettivo del Comitato Tecnico Scientifico del GARR partecipando attivamente ai progetti nazionali ed internazionali dedicati alle reti per la ricerca (N-REN) tra cui GARR-X Progress.}
	
	\ecvitem{2013--2016}{Delegato del Rettore all'ICT\newline
	Per il triennio 2013--2016, il Rettore dell’Università di Salerno, Prof. Aurelio Tommasetti, ha esteso la delega, nominandolo delegato del Rettore per l’ICT per l’Ateneo Salernitano.
	In tale ambito, il sottoscritto ha coordinato numerosi progetti di Ateneo dedicati alle infrastrutture tecnologiche tra cui la realizzazione di un nuovo sito Web di Ateneo, ed il progetto RIMIC per la realizzazione di una MAN regionale per l’interconnessione delle 7 università Campane. Per lo stesso periodo è stato rappresentante dell'Università di Salerno in seno al Consiglio di Amministrazione di RIMIC Scarl la società appositamente creata per la gestione dell’intera infrastruttura tecnologica realizzata con il progetto PON.}

	\ecvitem{1996--2016}{Laboratorio Specialistico Linux/TCFS e Sicurezza\newline
	Nell'ambito dello studio dei sistemi operativi e delle soluzioni per la sicurezza ha realizzato un laboratorio specialistico dedicato allo sviluppo di componenti del sistema (file system crittografico) ~\cite{USENIX:01} che nello spirito dell’Open Software sono state condivise da un'ampia comunità di utenti nel mondo. Nel tempo  hanno gravitato attorno al laboratorio ed alle attività ad esso connesse circa un centinaio di studenti, formando così un vero e proprio centro di competenza.}

	\ecvitem{2000--2003}{Progetto di ricerca “Oltre la Firma Digitale”\newline
	Nel 2000 ha istruito la presentazione di un progetto di ricerca nell’ambito della legge 297  finanziato dal MURST sui fondi della Legge 488 D.M. 629 per conto della società Sintel S.p.A. gruppo Finmatica. Obiettivo del progetto è stato lo sviluppo di un’infrastruttura a chiave pubblica per il superamento degli attuali limiti teorici della firma digitale~\cite{AICA:01} a supporto di soluzioni paperless.
	Successivamente il progetto triennale è stato finanziato dal Ministero per un importo di 9,089 Mld con decorrenza dal 1/10/2000 al 30/9/2003 e ne ha assunto la piena responsabilità in qualità di project manager.}

	\ecvitem{2005--2006}{Consulente Commissione Parlamentare di Inchiesta\newline
	Il sottoscritto ha partecipato alle attività della Commissione Parlamentare d’Inchiesta sugli effetti dell’impiego di uranio impoverito in qualità di consulente  esterno esperto informatico, partecipando in particolare, per quanto attiene le indagini, alla creazione di una base dati ed un campione statistico per l’analisi dei fenomeni rilevati sul campo.}

	\ecvitem{2005--2006}{Consulente CNIPA (Sicurezza ICT)\newline
	Nell’ambito del progetto per la Razionalizzazione del Pubblica Amministrazione Centrale, il sottoscritto ha collaborato con il Centro Nazionale per l’Informatica nella Pubblica Amministrazione alla realizzazione di un questionario (sezione sicurezza) compilato da tutte le PAC per la rilevazione dello stato delle varie Amministrazioni. Successivamente i risultati raccolti sono stati analizzati e presentati alle Amministrazioni per la definizione di un modello unico e condiviso per la sicurezza all’interno della PAC. Contemporaneamente sono state elaborate le linee guida per approntare nel breve le misure minime necessarie per il raggiungimento di un livello accettabile di sicurezza.}

	\ecvitem{2007--2009}{Consulente CNIPA (Sicurezza ICT e Continuità Operativa)\newline
	In occasione della stesura della relazione annuale (2006) sullo stato della Pubblica Amministrazione il sottoscritto ha rinnovato il contratto  di consulenza con il CNIPA per occuparsi direttamente della sezione dedicata alla Sicurezza ICT.
	Nel contempo nell’ambito del contratto si è occupato del tema Continuità Operativa con lo scopo di analizzare ed elaborare modelli per la C.O. per la PA da utilizzare per le principali amministrazioni della PAC.}

	\ecvitem{2006--2007}{Membro ordinario del Consiglio Superiore delle Comunicazioni\newline
	Con decreto del 22 Dicembre 2004 del Ministro elle Comunicazione On. M. Landolfi è stato nominato membro del Consiglio presidiato dall’avv. G. Massaro che si è insediato il 16 Marzo 2006 per il quadriennio 2006-2009. Nell’organigramma del Consiglio il sottoscritto partecipa alla terza ed alla quarta sezione, aventi rispettivamente come attribuzioni :
	Ricerca e sperimentazione, nuove tecnologie, istruzione ed aggiornamento professionale, la prima, e Multimedialità ed intermedialità, contenuti affari non suscettibili di rientrare nella competenza delle altre sezione o della giunta, la seconda. Il Consiglio, che rappresenta il massimo organo consultivo del Ministro, si riunisce con cadenza bisettimanale per il necessario supporto alle attività del Ministero delle Comunicazioni. In funzione del D.P.R. del 14 maggio 2007, n. 90, è decaduto dalle funzioni.}

	\ecvitem{2009--2012}{Membro del CTS della Provincia di Salerno}
	
	\ecvitem{2011--2017}{Membro dell’Ufficio Piano di Zona Ambito SA5, Salerno\newline
	Nell’ambito della struttura dedicata ai servizi sociosanitari del Comune di Salerno e del Comune di Pellezzano partecipa al tavolo istituzionale in qualità di esperto informatico per le attività di informatizzazione dei servizi erogati.}
}


\ecvtitle{}{Progetti, collaborazioni e convenzioni}

	Il sottoscritto è stato responsabile o proponente di numerosi progetti di ricerca in ambito ICT condotti in partnership con importanti aziende o con Enti Pubblici attraverso convenzioni stipulate con il DIA. Tra queste si citano le più recenti e significative:
	
	\ecvitem{2005--2008}{Responsabile scientifico accordo quadro con Provincia di Salerno, Salerno\newline
	In collaborazione con il prof. A. De Santis e di un gruppo di giovani ricercatori è stato sviluppato un sistema di comunicazione per dispositivi mobili (GSM) in grado di garantire la massima privacy della conversazione ed il non ripudio (firma digitale). Il progetto  Speech~\cite{Castiglione2006287} è stato finanziato dalla Provincia di Salerno attraverso un contributo di di 80.000~\euro{}. Dopo le fasi di progettazione, è stato sviluppato un prototipo dimostrabile che è stato presentato a numerose aziende del settore delle telecomunicazione. Tali risultati sono stati unanimemente ritenuti estremamente significativi ed innovativi per un settore estremamente delicato da affrontare in  maniera laica (quello delle intercettazione telefoniche). La Provincia di Salerno, nel riconoscimento dei risultati raggiunti, ha offerto il suo contributo per promuovere la fase di pre-industrializzazione. Tali risorse sono state utilizzate per allestire un laboratorio avanzato sul tema comunicazioni e privatezza che attualmente opera nel settore delle comunicazioni VoIP.}

	\ecvitem{2009--2013}{Responsabile unità locale Progetto MISE Made in Italy "OPEN"\newline
	Nell'ambito del bando di finanziamento del MISE denominato {\em Made in Italy} ha partecipato alla presentazione del progetto OPEN, successivamente ammesso a finanziamento per un importo complessivo di 6,4 M\euro{}. Il progetto ha raccolto in ATS numerosi partner nazionali tra cui la società di consulenza Everis Italia S.p.A. (capofila) ed aziende private quali: Prima Electro di Moncalieri (TO), Domini Officine di Alba (CN),  Taglio Srl di Piobesi d’Alba (CN), ERXA di Torino, Motor Power Company di Castelnovo di Sotto (RE). Il progetto ha avuto come obiettivo la realizzazione di una macchina a controllo numerico per il taglio (ottimizzato) ad acqua delle pelli. Il sottoscritto ha diretto l'unità salernitana (il DIA è stato partner del progetto in ATS), che ha contribuito sia sugli aspetti di sicurezza che sugli sistemistici affrontati per la remotizzazione delle primitive per il controllo delle attrezzature sul campo. Il progetto si è concluso con successo nel 2013 dopo 36 mesi di attività e l'unità salernitana ha ottenuto un finanziamento di 185.000 \euro{}.}

	\ecvitem{2012--2015}{Responsabile OR Progetto Campus Salus per Lactem, Salerno\newline
	Nell'ambito del bando della Regione Campania per la concessione di aiuti a progetti di ricerca industriale e sviluppo sperimentale per la realizzazione di campus dell’innovazione in attuazione delle azioni  a valere sugli obiettivi operativi 2.1 e 2.2. del POR Campania 2007/2013, in collaborazione con il Dipartimento di Farmacia (Capofila per l'Università di Salerno) ed il Dipartimento di Ingegneria Elettronica oltre ad un gruppo di aziende tra cui la capofila Centrale del Latte di Salerno, ha presentato un proposta progettuale finalizzata al tracciamento ed al recupero del serio caseario, denominata {\em Salus per Lactem}. Il progetto industriale molto ambizioso è stato affrontato introducendo le soluzione più moderne esistenti in letteratura per disporre di dati raccolti direttamente presso le stalle al momento della raccolta. Un gruppo di sensori appositamente realizzati hanno infatti permesso di isolare parametri qualitativi e quantitativi del latte raccolto (come acidità, carica batterica, quantità di grassi). \`E stato poi successivamente dimostrato che tali parametri risultano fortemente caratterizzanti per quanto riguarda le proprietà organolettiche del prodotto finale e dei suoi derivati (yogourt, formaggi, prodotti salutistici, ecc.). Il progetto è stato finanziato per un costo complessivo di 3,6 M\euro{}.
	Il sottoscritto, responsabile del Obiettivo Realizzativo 2 (Progettazione del sistema di tracciabilità), ha realizzato insieme all'azienda NexSoft tutto il sistema per la raccolta dati ed il tracciamento delle materie prime.
	Il progetto ha concluso le attività di disseminazione dei risultati nell'ottobre del 2015.}

	\ecvitem{209--2021}{Responsabile dell'accordo quadro tra Università di Salerno e la Polizia Postale\newline
	Il gruppo di ricerca ha collaborato con il CNCPO (Centro Nazionale per il Contrasto alla Pedopornografia Online), organo della Polizia di Stato, al fine di individuare, sviluppare e valutare sperimentalmente soluzioni altamente innovative capaci di affrontare il tema della \emph{source camera identification}, applicato in maniera massiva a immagini provenienti dalla rete o dai cosiddetti online social networks. Le soluzioni sviluppate sono condivise con gli esperti della Polizia di Stato ed utilizzate sia per le indagini che in ambito forense.
	}
  
  \ecvsection{Formazione}
  
  \ecvtitlelevel{1978--1983}{Laurea in Scienze dell'Informazione}{}
   \ecvitem{} {Si è laureato con \emph{lode} in Scienze dell'Informazione presso l'Università di Salerno discutendo una tesi dal titolo {\em “Architetture Special Purpose per l'Elaborazione di Immagini”}.}
  
  \ecvtitle{1973--1978}{Maturità Scientifica}
   \ecvitem{} {Ha ottenuto la maturità scientifica presso il Liceo Scientifico F. Severi, Salerno con votazione \emph{60/60}}
%  \ecvitem{}{Brunel University, London United Kingdom}
%  \ecvitem{}{
%      \begin{ecvitemize}
%	\item sociology of risk
%	\item sociology of scientific knowledge / information society
%	\item anthropology
%	\item E-learning and Psychology
%	\item research methods
%      \end{ecvitemize}
%  }
%  
%  \pagebreak
 
  \ecvsection{Personal skill}
  \ecvmothertongue{Italiano}
  \ecvlanguageheader
  \ecvlanguage{Inglese}{B2}{C1}{C1}{C1}{C1}
  \ecvlastlanguage{Francese}{C1}{C2}{C1}{C1}{C1}
  \ecvlanguagefooter
   
  \ecvblueitem{Communication skills}{
  \begin{ecvitemize}
    \item team work: Ottima capacità di team building, ha fatto parte di numerosi gruppi di lavori e squadre di atleti
    \item mediating skill: Ottima capacità di ascolto e di negoziazione in condizioni di stress psico fisico.
    \item intercultural skill: Buona capacità di integrare culture ed approcci diversi.
  \end{ecvitemize}
  }
  
  \ecvdigitalcompetence{\ecvProficient}{\ecvProficient}{\ecvIndependent}{\ecvProficient}{\ecvProficient}
  
  \ecvblueitem{Computer skill}{
  \begin{ecvitemize}
    \item Esperto dei principali linguaggi di programmazioni: Lisp, C, C++, Java, Scala, Python
    \item Esperto dei principali framework di sviluppo: Visual Studio, Netbeans, Eclipse, IntelliJ
    \item Esperto dei sistemi operativi: Unix, Linux, Mac OS X, Windows.
    \item Esperto di calcolo distribuito in particolare secondo il paradigma del MapReduce negli ambienti Apache Hadoop e Spark.
    \item Esperto delle principali librerie per il Machine Learning, in particolare di Apache MLLib nel distribuito e di Scikit, TensorFlow, Keras, PyTorch in ambiente Python. 
  \end{ecvitemize}
  }
  
  
  \ecvblueitem{Other skill}{
    \begin{ecvitemize}
		\item Appassionato di mare, vela e sport nautici in generale
        \item Pratica Tennis Tavolo a livello agonistico e fitness
	  \end{ecvitemize}
	}
  % \ecvblueitem{Driving licence}{A, B}
  
  
  \nocite{*}
%  \ecvsection{Pubblicazioni}
  
  
  %\ecvblueitem{}{\nocite{*}}
%  \textit{How to do Observations: Borrowing techniques from the Social Sciences to help Participants do Observations in Simulation Exercises}, Coyote EU/CoE Partnership Publication, (2002).
  % \newpage
%  \renewcommand\refname{Bibliografia}
%  \printbibheading
  %\printbibliography[prefixnumbers={A},type=article,title={Articoli su riviste internazionali},heading=subbibliography]
  %\printbibliography[prefixnumbers={C},type=inproceedings,title={Articoli a conferenze},heading=subbibliography]
  
  
%  \begin{refcontext}[labelprefix=J]{rc}
%  \renewcommand*{\bibfont}{\footnotesize}
%  \printbibliography[type=article, title={Articoli su riviste internazionali}, heading=subbibliography, resetnumbers=1]
%  
%  \end{refcontext}
%  
%  
%  \begin{refcontext}[labelprefix=W]{rc}
%  \renewcommand*{\bibfont}{\footnotesize}
%  \printbibliography[nottype=article,title={Articoli presentati a conferenze internazionali}, heading=subbibliography, resetnumbers=1]
%  
%  \end{refcontext}
%  
\renewcommand{\section}[2]{\ecvsection{#2}}
\renewcommand{\subsection}[2]{\ecvsection{#2}}

\printbibtabular[type=article, title={Articoli su riviste internazionali}, heading=subbibliography, resetnumbers=1]
  
  
\printbibtabular[nottype=article, title={Articoli presentati a conferenze internazionali}, heading=subbibliography, resetnumbers=1]



\ecvsection{Trattamento dei dati personali}
  
\ecvitem{}{\emph{Autorizzo il trattamento dei dati personali contenuti nel mio  Curriculum Vitae ai sensi dell’art. 13 Dlgs 196 del 30 giugno 2003 e  dell’art. 13 GDPR (Regolamento UE 2016/679).}}


\bigskip

Salerno, \currentDate

\hfill\parbox[t]{8cm}{
  \begin{center}
    firmato\\
    \textit{Giuseppe CATTANEO}
	 \includegraphics[width=0.9\linewidth]{../../Modelli/Signatures/cattaneosignature}
  \end{center}
%  \hrule   
}
	

  \clearpage
  
  \end{europasscv}

\end{document}