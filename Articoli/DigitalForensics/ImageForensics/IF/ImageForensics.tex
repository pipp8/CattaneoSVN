%
% $Header:  $
%
%\documentclass{acm_proc_article-sp}
\documentclass[runningheads,11pt]{llncs}
\usepackage[hmargin=1.5in,vmargin=1.8in]{geometry}
\usepackage{graphicx}


\newcommand{\Owner}{\ensuremath{\mathsf{O}}}
\newcommand{\SM}{\ensuremath{\mathsf{SM}}}
\newcommand{\User}{\ensuremath{\mathsf{U}}}
\newcommand{\Proxy}{\ensuremath{\mathsf{P}}}
\newcommand{\SC}{\ensuremath{\mathsf{SC}}}
\newcommand{\RSC}{\ensuremath{\mathsf{RSC}}}
\newcommand{\SRSC}{\ensuremath{\mathsf{SRSC}}}
\newcommand{\PRSC}{\ensuremath{\mathsf{PRSC}}}
\newcommand{\MRSC}{\ensuremath{\mathsf{MRSC}}}
\newcommand{\PSCS}{\ensuremath{\mathsf{PSCS}}}

\newenvironment{myitemize}
{\begin{list}{$\bullet$}{ \itemindent=-.663cm \listparindent=.6cm
\itemsep=0.0in
\parsep=0.0in
\topsep=0.0in
\partopsep=0.0in}}{\end{list}}

\newcounter{itemcount}
\newenvironment{myenumerate}
{\setcounter{itemcount}{0}\begin{list}
{\arabic{itemcount}.}{\usecounter{itemcount} \itemindent=0.0cm
\itemsep=0.0in
\parsep=0.0in
\topsep=0.0in
\partopsep=0.0in}}{\end{list}}

\newenvironment{mydescription}
{\setcounter{itemcount}{0}\begin{list}
{\arabic{itemcount}.}{\usecounter{itemcount} \itemindent=0.0cm
\itemsep=0.0in
\parsep=0.0in
\topsep=0.0in
\partopsep=0.0in}}{\end{list}}


\newcommand{\remove}[1]{}


\begin{document}
\title{Improved Image Source Analysis}
\titlerunning{Improved Image Source Analysis}
%
%\numberofauthors{1}
\author{ Pippo \and Pluto \and Paperino}
\authorrunning{ short author list}
\institute{ Dip. di Informatica ed Applicazioni {\sl R.M. Capocelli}\\
   University of Salerno - ITALY\\
   \email{\{pippo,pluto,paperino\}@dia.unisa.it}
}

\maketitle
\begin{abstract}

In this paper, we present an improvement over the digital camera
identification technique based on sensor noise proposed by Lukas
\emph{et al.}\cite{Fri06}.

The project is a joint work with the C.N.C.P.O.  (Centro Nazionale
per il Contrasto della Pedopornografia sulla rete internet,
Dipartimento della Pubblica Sicurezza del Ministero dell'Interno, by
the Italian Servizio Polizia Postale e delle Comunicazioni). Our
main goal is to develop a a full fledged system prototype (a tool
set supported by a robust methodology) able to give an answer to the
following question: {\em which camera among a predefined set could
be the source of a given digital image ?} This was formerly known as
the problem of the identification of how a given image has been
generated). As this method as been conceived as a digital image
forensics technique to support policy investigation, the result is
acceptable only if it comes with the highest degree of accuracy.

The project started from the results presented in \cite{Fri06} which
use the characteristic noise of the sensor, typically a CCD
(charge-coupled device), as a fingerprint to identify the source
camera. We first implemented a framework to perform an experimental
analysis of such a technique and to verify on the field if it could
be successfully used for our specific purposes with the necessary
reliability. During this phase we stressed the original technique
with a big amount of consumer and professional digital cameras
populating our data base with several thousand of digital images.
All the test confirmed that the pixel non-uniformity noise could be
successfully used to identify the image source but for some specific
camera models we got a sort of pattern clash and many images
produced low correlation values with more than one reference pattern
previously extracted from the cameras under investigation including
the true source. In these cases the values were also too close to
the rejection threshold. This has been considered an important issue
to adopt the Fridrich's approach for digital image forensics.

Moreover the process to define the acceptance threshold based on the
Neyman-Pearson approach, which has been originally employed to
define the acceptance threshold $ t $ minimizing the false reject
rate (FRR), resulted too time consuming e not enough flexible
because the necessary accuracy can be reached only if the system can
learn from the analysis of a considerable number of images from  a
known source and when the camera set or the input image set change
all the threshold $ t $ must be recomputed.

In this paper we present three variants to the original technique
based on a different approach in the classification phase to
identify the source camera. We first define a set $ S $ of candidate
digital cameras, than we  extract the sensor noise reference pattern
for each camera. Then we take a small set (even a dozen) of randomly
chosen pictures for each camera. For each image we calculate the
correlation factor against the reference pattern of each camera.
Finally we use these values as training set for a SVM classifier
(Support Vector Machine) with a number of input attributes equal to
the cardinality of the set $ S $. Similarly we defined a test set to
verify the result of the training. The trained classifier is now
ready to  answer to the question \emph{which camera among the set $
S $ produced the image $ I $ ?} In fact we recalculate the
correlation with all the reference patterns and we give the
resulting values as input to the SVM which can provide us with the
source camera identification.

We have extensively tested the approach we propose and, as shown in
the section \ref{sec:experiments}, it led us to a double
improvement:

\begin{myenumerate}
\item on one hand our approach improved the accuracy of the final
result especially in the case of cameras with sensor noise patterns
which produces low correlation values or simply too close to be
considered distinct. In facts, the SVM can identify the source
discriminating these values considering the correlation values
produced by the patterns of the other cameras of the set $ S $
instead of comparing each correlation value against a single
predefined threshold value.


\item on the other hand our approach produced a considerable speed up
of the whole process for source camera identification, reducing the
amount of the images necessary to train the system establishing the
threshold of acceptance. This becomes even more meaningful if is
considered that during the investigation often we had to work with a
set of hundred candidate cameras, and many of these belong to the
same producer / model.
\end{myenumerate}
\end{abstract}

%\keywords{Smart Card, Proxy Cryptography, Network Security Appliance}

\section{Introduction}

\section{System Components}

Comparing our system with the original one proposed by Lukas
\emph{et al.} \cite{Fri06} three main differences arise:

\begin{enumerate}
\item The process to extract the camera reference pattern is based
on a series of images with a very smooth subject, such as a white
and uniformly lighted wall. Moreover the process for pattern
extraction has been logically separated by the process for source
camera identification. More in details, for each camera $ k $  we
produce a set of $ N^k $ template images $ T $, and for each image $
I_x, x = 1, \ldots, N^k $ we apply a filter $ F $ to remove the
scene content leaving only high frequency noise. The final camera
reference pattern $ np^k $ is obtained by averaging the residual
noise pattern as shown in \ref{form:pat}

\begin{equation}
np^k = AVG_x(n_x = I_x - F(I_x)) \label{form:pat}
\end{equation}

\item The denoising filter $ F $ is based on our implementation of
a flexible wavelet package able to efficiently manage images of any
size (using padding to a smallest power of 2) (with 8 or 16
coefficients). In this way ...

\item The identification process is based on a trained SVM. The target
input image $ I^T$ is filtered with the same filter $ F $ to obtain
the image noise pattern $ np^{I^T} $. Then, for each camera $ k $ we
calculate a correlation value between the image noise pattern $
np^{I^T} $ and the camera noise pattern $ np^k $:

\begin{equation}
\rho^{k} = corr( np^{I^T}, np^k), k = 1, \dots, N \label{form:np}
\end{equation}

the resulting ordered data set (just one raw) becomes the SVM input,
that will give the final answer providing the user with the index $
k $ of the camera whose reference pattern is closest to the image
pattern. Obviously this answer will consider the global distance,
considering also the relations of the chosen reference pattern with
all the other cameras reference pattern.
\end{enumerate}

We will demonstrate how item 1 produces better reference patterns,
comparing the correlation values between generic input images and
our reference pattern with the ones obtained using reference pattern
extracted from generic images.

Item 2 radically reduced the use of cropping or resizing operator,
giving again higher correlation values. Moreover this makes the
system suitable to efficiently handle real word images.

We will demonstrate how item 3 improved the performance of the
overall process reducing the classification errors.


\section{Experimental Analysis}
\label{sec:experiments}

\begin{tabular}{|l|c|c|}
  \hline
  Camera model & \# identification error & \# identification  error \\
               & Fridrich's method    & SVM classifier \\
  \hline
  % after \\: \hline or \cline{col1-col2} \cline{col3-col4} ...
  Canon 400D  & 2 & 0 \\
  Canon $ A400_1 $ & 8 & 2 \\
  Canon $ A400_2 $ & 1 & 0 \\
  HP E327 & 5 & 0 \\
  Kodak CX7530  & 1 & 0 \\
  \hline
\end{tabular}
\section{Conclusion}



\bibliographystyle{acm}
\bibliography{ImageForensics}

\end{document}
